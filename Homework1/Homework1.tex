\documentclass{article}
\usepackage{graphicx} % Required for inserting images
\usepackage{amsmath}
\usepackage[margin=.5in]{geometry}


\title{Homework 1}
\author{Quan Nguyen}
\date{January 2026}

\begin{document}

\maketitle

\section{Airplane seat overselling}.

\subsection*{a} 

Looking at this question, the binomial distribution represents the theoretical distribution 
since we want to count the amount of sucesses 
within a certain amount of samples. However, if we count a passenger missing a flight being a success with probability $p = .02$,
with a large sample of $n=100$ and an expected value of $np = 2$. We can approximate it with
a poisson distribution where $\lambda = np = 2$

The probability of not having enough seats occurs when the event of only 0 or 1 passengers misses their flight.
\begin{equation}
    P(X\leq 1) = P(X=0) + P(X=1) = \frac{2^0e^{-2}}{0!} + \frac{2^1e^{-2}}{1!} = .4060
\end{equation}

There is a $.4169$ probability that there will not be enough seats on the plane.

\subsection*{b} 
Binomial distribution pdf is $\binom{n}{k}(1-p)^{n-k}p^k$ where $n = 100$ and $p = .02$
\begin{align}
    P(X = 0) &= \binom{100}{0}(1-p)^{100}p^0 = 1*.98^{100} = .1326\\
    P(X = 1) &= \binom{100}{1}(1-p)^{99}p^1 = 100*.98^{99}*.02 = .2707\\
    P(X \leq 1) &= P(X=0) + P(X=1) = .4033\\
\end{align}

The approximation using the poisson distribution is slightly larger then the binomial distribution
but is overall pretty close.
\subsection*{c} 
This is a conditional probabiliy where given the plane has all its passenger, there are exactly two no shows.
The probability of 2 no shows and the plane being full can be modeled with a binomial distribution in equation $(6)$.
The probability that every seat is filled occurs when there are two or less no shows which is modeled in equation $(7)$. The conditional probability is calculated in equation $(8)$.
The results indicate a probability of .4040 that an airline will not need to reimburse passengers given the plan is full.
\begin{equation}
    P(X\geq 2 \cap X \leq 2) = P(X = 2) = \binom{100}{2}(1-p)^{98}p^2 = \frac{100!}{2!98!}.98^{98}.2^2 = .2734
\end{equation}
\begin{equation}
    P(X \leq 2) = P(X = 2) + P(X\leq 1) = .2734 + .4033 = .6767
\end{equation}
\begin{equation}
    P(X\geq 2 \cap x \leq 2|X\leq 2) = \frac{P(X=2)}{P(X\leq2)} = \frac{.2734}{.6767} = .4040
\end{equation}

\newpage
\section{Joint distributions}
\begin{equation}
    f(x,y) = \begin{cases}
        Cxy & 0\leq y \leq x, 0 \leq x \leq 1\\
        0 & \text{otherwise}
    \end{cases}
\end{equation}

\subsection*{a}
We need to find some c where the joint cummulative density equals 1 when the entire sample space is integrated.
\begin{align}
    1 = \int_{-\infty}^{\infty}\int_{-\infty}^{\infty} Cxy dR&= 0 + \int_{0}^{1}\int_{0}^{x} Cxy dydx\\
    &= \frac{1}{2}\int_{0}^{1}Cx^3dx\\
    &= \frac12\frac14 cx^4|_0^1 = C\frac{1}{8}\\
    &\implies c = 8
\end{align}

Hence $C = 8$ for the joint distribution

\subsection*{b}
The joint distribution is not independent. By looking at it, when x, is 0, the probability of get 0 for y is $100\%$,
but clearly when x is some other value, the probability of y being 0 is not 1. This can be proven more rigorously by showing 
the product of the marginal distribution does not equal the joint distribution.

\begin{align}
    f_x = \int_{0}^{x} 8xy dy = 4x^3\\
    f_y = \int_{y}^{1} 8xy dx = 4(y-y^3)\\
    f_x*f_y = 8(x^3y - x^3y^3) \neq 8xy
\end{align}

\subsection*{c}

\newpage

\section{Numerical evaluation of integrals via Monte Carlo simulation}
Consider the following equation
\begin{equation}
    I = \int_1^\infty \frac1{1+x^6}dx
\end{equation}

\subsection*{a}
I will be using $u = \frac{1}{x}$ and $du = \frac{-1}{x^2}dx$ for the substitution yielding:
\begin{align}
    \int_1^0 \frac{1}{1+\frac{1}{u^6}}*-x^2du\\
    \int_0^1 \frac{1}{u^2 + \frac{1}{u^4}}du\\
    \int_0^1 \frac{u^4}{u^6+1}du\\
\end{align}

Note that 


\end{document}