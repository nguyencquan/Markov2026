\documentclass{article}
\usepackage{graphicx} % Required for inserting images
\usepackage{amsmath}
\usepackage[margin=.5in]{geometry}
\setlength{\parindent}{0pt}

\title{Homework 3}
\author{Quan Nguyen}
\date{February 2026}

\begin{document}
\maketitle
See https://github.com/nguyencquan/Markov2026 for code used in simulations.
\section{Acceptance–Rejection with Optimized Proposal}
\textbf{a)}
We will begin by dividing the two equations resulting in
\begin{align}
    \frac{f(x)}{g_a(x)} &= \frac{x(1+x)e^{-x}}{3a^2xe^{-ax}}\\
    &= \frac{(1+x)e^{(a-1)x}}{3a^2}\\
    &=\frac{(1+x)}{3a^2}\frac{1}{e^{-(a-1)x}}\\
\end{align}
Make a note that the rate paramemter must be positive and that x is always positive
or 0. Hence $\frac{(1+x)}{3a^2}$ is always positive and $e^{(a-1)x}$ is also positive.
If we take the limit of the quotient, it is an undetermined form that can be converted
using L'H resulting in 
\begin{equation}
    \frac{1}{3a^2}\frac{1-a}{e^{-(a-1)x}}
\end{equation}
In order for the equation to not be unbounded, $a-1<0$ to keep the exponential
value in the denominator. Since the equation is continuous and $x = 0$ does not diverge
the equation must contain an absolute maximum if $0<a<1$.

\textbf{b)}

We will now find some value of a where x is maximized by taking the derivative and
solving for 0. 
\begin{align}
    \frac{dc}{dx} &= \frac{d}{dx}\frac{(1+x)e^{(a-1)x}}{3a^2}\\
    &=\frac{e^{(a-1)x}}{3a^2}+\frac{(1+x)(a-1)e^{(a-1)x}}{3a^2}\\
    &=\frac{1}{3a^2}(a+ax-x)e^{(a-1)x}\\
    \implies 0&= (a+ax-x)\\
    0&= a+x(a-1)\\
    -a& = x(a-1)\\
    x &= \frac{a}{1-a}
\end{align}
Note $\lim_{x\rightarrow \infty} h_a(x) = 0$ and $h_a(0) = \frac{1}{3a^2}>0$. Since there is only
one critical point, the function is always positive, and it is continuous, the critical point must be a maximum. To
calculate the actual maximum, simply substitute.
\begin{align}
    c(a) &= \frac{(1+\frac{a}{1-a})e^{(a-1)\frac{a}{1-a}}}{3a^2}\\
    &= \frac{\frac{1}{1-a}}{3a^2e^a}\\
    &= \frac{1}{{3a^2e^a(1-a)}}\\
\end{align}

\textbf{c)}

This is an optimization question. Furthermore $1/c(a)$ is the lowest probability of acceptence hence 
we should look to maximize it with some value a. Simply take the derivative and solve for 0.
\begin{align}
    \frac{d}{da} 3a^2e^a(1-a) &= 6ae^a(1-a) + 3a^2e^a(1-a) - 3a^2e^a\\
    &= (6a-6a^2 + 3a^2-3a^3 - 3a^2)e^a\\
    &= 3(2a-2a^2-a^3)e^a\\
    \implies 0 &= -a^3-2a^2+2a\\
    0 &= -a(a^2+2a-2)\\
    \implies a &= 0, \frac{-2\pm \sqrt{4+8}}{2}\\
    &=0,-1 + \sqrt{3},-1 - \sqrt{3}\\
\end{align}

However since $0<a<1$ the only viable value for a is $a = \sqrt3-1 \approx .7321$. To make sure this value does
maximize $1/c(x)$ we must take the second derivative.
\begin{align}
    \frac{d}{da}3(2a-2a^2-a^3)e^a|_{a = \sqrt3-1} &= 3(2a-2a^2-a^3)e^a + 3(2-4a-3a^2)e^a|_{a = \sqrt3-1}\\
    &=3(2-4a-3a^2)e^a|_{a = \sqrt3-1}\\
    &= -15.81899 < 0
\end{align}
Since the second derivative is negative, the point $a = \sqrt 3 - 1$ is concave down and is hence the maximum.
This results in:
\begin{equation}
    P(\text{Acceptance}) = 3(a^*)^{2}e^{a^*}(1-a^*) = .8957
\end{equation}
\textbf{d)}

\begin{figure}[h!]
    \centering
    \includegraphics[width=0.7\linewidth]{Q1pdf.png}
    \caption{Graph showing the pdf for $f(x)$ and $c(a^*)g_{a^*}(x)$}
\end{figure}

\textbf{e)}
If there is no program that can sample from the exponential distribution,
use inverse sampling by calculating the CDF of the exponential distribution:
\begin{align}
    F(X) &= \int_0^x a^*e^{-a^* x}dx\\
    & = -e^{-a^* x}|_0^x\\
    & = 1-e^{-a^*x}\\
\end{align}

Then we take the inverse function which is
\begin{equation}
    x = -\frac{\ln(1-k)}{a^*}
\end{equation}

To sample from the exponential distribution, sample $k\sim U(0,1)$ and plug it into
the inverse function above. 

However to calculate $g_{a^*}(x)$. Use inverse sampling method twice to yield
$x_1$ and $x_2$. Sum those two values to obtain a sample from the gamma distribution
$g_{a^*}(x)$

\section{Cell Cycles}

\textbf{a)}

First we will define states:

1. G is state 1

2. S is state 2

3. D is state 3

Hence we get a probability transition matrix of
\begin{equation}
    p = \begin{bmatrix}
        \dfrac{9}{10} & \dfrac{1}{10} & 0 \\[10pt]
        0 & \dfrac{7}{8} & \dfrac{1}{8} \\[10pt]
        \dfrac{2}{5} & 0 & \dfrac{3}{5}
        \end{bmatrix}
\end{equation}

\textbf{b)}
By modeling it with a geometric distribution, we can set p
as the probability that the cell leaves to state S which is $p = 1/10$.
The expected value for a geometric distribution is $1/p = 10$.
Hence the expected time the cell stays in the growing state is 10,

\textbf{c)}
Note that for each state, it can stay in it's state or go forward, but it cannot go to a previous state.
In other words, $p_{13} = p_{21} = p_{32} = 0$. Hence if we sum up how long each state lasts for,
since it can only go forward, we can calculate one cycle. Also if we treat each hour like a bernouli
random variable, then leaving can be modeled with a geometric distribution with p being the probability
of leaving to the next state. By taking the expected value, we get the time in one state. Sum up all these time to get the
expected time for one cycle
\begin{align}
    p_{12} &= \frac 1{10} \implies E[Geo(p_{12})] = 10\\
    p_{23} &= \frac 1{8} \implies E[Geo(p_{23})] = 8\\
    p_{31} &= \frac 2{5} \implies E[Geo(p_{31})] = \frac{5}{2}\\
    10&+8+2.5 = 20.5
\end{align}

Hence the expected duration of one cycle is $20.5$ hours.

\textbf{d)}
The expected time spent in G state is 10 hours while a cycle is expected to last for 20.5
hours so each cell is expected to spend $10/20.5 = 48.78\%$ of the time in the G state.

\textbf{e)}
Using r, I multiplied the transition matrix together 50 times. The resulting matrix:
\begin{equation}
    p^{50} = \begin{bmatrix}
        0.4878& 0.3902 &0.1220\\
        0.4878& 0.3902 &0.1220\\
        0.4878& 0.3902 &0.1220\\
    \end{bmatrix}
\end{equation}
This shows the probability of what state the cell is in after 50 cycles. Notice how the first column
matches the calculated probability of $.4878$ of time spent in G state. Also since all values in the column
are the same, final position is independent of starting position.

\textbf{f)}
This will be done by iterating 1000 times a markov chain and storing how often a state appears.
For each state, calculating the next state is done through inverse discrete sampling. See code and code 
comments for more detail. After running the markov chain, the chain is in a state of G $4732$ times, which
is a $.4732$ probability at any time, a value close to the first column of $p^{50}$.

\section{Markov Chain 1}
\textbf{a)} Note the total amount of steps must sum up to 3 to go from 2 to 5 hence we can take the permutation for
$(1,1,1,1,-1)$ resulting in the possible steps with conversion to states being
\begin{align}
    (1,1,1,1,-1) &\rightarrow (2,3,4,5,5,5)\\
    (1,1,1,-1,1) &\rightarrow (2,3,4,5,5,5)\\
    (1,1,-1,1,1) &\rightarrow (2,3,4,3,4,5)\\
    (1,-1,1,1,1) &\rightarrow (2,3,2,3,4,5)\\
    (-1,1,1,1,1) &\rightarrow (2,1,1,1,1,1)\\
\end{align}
Hence we end up with the only pathways being the following sequence of steps and probabilities:
\begin{align}
    (1,-1,1,1,1) &\implies (\frac23)^4 + \frac13=.0659\\
    (1,1,-1,1,1)&\implies (\frac23)^4 + \frac13=.0659\\
    (1,1,1,0,0)&\implies (\frac23)^3= .2963\\
\end{align}

By summing up these values we get a total probability of $.428$ of being in state 5 after 5 steps.

\textbf{b)}
The row indicate the starting position and column final:
\begin{equation}
    p^5 = \begin{bmatrix}
        1&0&0&0&0\\
        \frac13&0&\frac23&0&0\\
        0&\frac13&0&\frac23&0\\
        0&0&\frac13&0&\frac23\\
        0&0&0&0&1
    \end{bmatrix}
\end{equation}



\textbf{c)}
We wil be calculating $p^5$ instead of $p^4$ as we are doing 5 steps and not 4.
\begin{equation}
    p = \begin{bmatrix}
        1&0&0&0&0\\
        .4403&0&.1317&0&.4280\\
        .1605&.0658&0&.1317&.6420\\
        .05350&0&.0658&0&.8807\\
        0&0&0&0&1
    \end{bmatrix}
\end{equation}
Furthermore:
\begin{equation}
    P(X_5 = 5|X_0 = 2) = p^5_{2,5} = .428
\end{equation}
Hence using matrix multiplication and summing all possible events equaled each other which is expected.

\textbf{d)}

We will consider $V_i = P(\text{starting at state i and ending up in state 1 without going to 4})$
Solving for $V_3$ we have
\begin{align}
    V_2 &= \frac13 + \frac23 V_3\\
    V_3 &= \frac13 V_2 + \frac23 V_4\\
\end{align}
We will be treating state 4 as an absorbing state since if we go to 4 we don't want to be able to go back to 1. Hence
$V_4 = 0$ and using algebra:
\begin{align}
    V_3 &= \frac13 (\frac13 + \frac23 V_3)\\
    V_3 &= \frac19 + \frac29 V_3\\
    7V_3 &= 1\\
    \implies V_3 &= \frac17\\
\end{align}
Hence the probability of starting at 3 and ending in state 1 without going through state 4 is $.1429$.
\section{Markov Chain 2}
\textbf{a)} The way I did it is by drawing out the markov chain to double check. However by quickly looking at it
state 5 and 6 loop with each other forever and are hence recurrent. Eventually, all the other states willl end up in 5
hence every single other point is transient. Also 5 and 6 are in closed communicating classes whille the other points are in an open communicating classes
hence using the conjecture from class:

Recurrent States: 5,6

Transition States: 1,2,3,4

\begin{figure}[h!]
    \centering
    \includegraphics[width=0.7\linewidth]{makrovChain2.png}
    \caption{Markov chain and communicating classes}
\end{figure}

\textbf{b)} Using the theorem from class and r to multiply the matrices together...
\begin{equation}
    P(X_5 = 4|X_0 = 1) = p^{5}_{1,4} = .1759
\end{equation}

\textbf{c)} Again using the same strategy where each row is a stochastic matrix that can be used
to calculate the CMF. Then discrete inverse sampling is used to calculate the position. After running 
the sample 10000 times, we see that the final state of 4 occurs 1781 times which is a probability of $.1781$ similar to the value calcualted in 3.b.
\newline
\newline
\textbf{d)}
We can calculate this by simply conditioning each event by a future event.  To solve $p_{3,3}$ we will define:
\begin{equation}
    V_{i} = P(\text{Starting at i and ending up in 3})
\end{equation}
\begin{align}
    V_{1} &= \frac12V_{1} + \frac12V_{2}\\
    V_{2} &= \frac12V_{2} + \frac12\\
    V_{3} &= \frac13V_{1} + \frac13 + \frac13V_{4}\\
    V_{4} &= 0\\
    V_{5} &= 0\\
    V_{6} &= 0\\
\end{align}
Note $V_5,V_4,V_6 = 0$ due to 5 and 6 being in a closed communicating class and $p_{4,3} = 0$ since
there is no way to go back. Note also $p_{4,1},p_{4,2},p_{4,3}=0$.
By doing some algebra:
\begin{align}
    \frac {V_2}2 &= \frac12\\
    V_2 &= 1\\
    \implies \frac{1}{2}V_1 &= \frac12\\
    V_1 &= 1
\end{align}
Hence:
\begin{equation}
    p_{3,3} = V_3 = \frac13 * \frac{1}{2} + \frac13 + \frac13 *0= \frac12
\end{equation}

To solve $p_{1,1}$ we will define:
\begin{equation}
    V_{i} = P(\text{Starting at i and ending up in 1})
\end{equation}
We will also use the fact that state 4 has no way of going to 1,2 or 3. Hence we have the system
\begin{align}
    V_1 &=\frac12 + \frac12 V_2\\
    V_2 &= \frac12V_2 + \frac12V_3\\
    V_3 &= \frac13 + \frac13V_3\\
\end{align}
Hence through algebra:
\begin{align}
    \frac23 V_3 &= \frac13\\
    V_3 &= \frac12\\
    \implies \frac12 V_2 &= \frac14\\
    V_2 &= \frac{1}{2}\\
    \implies p_{1,1} = V_1 &= \frac12 + \frac12\frac12 = \frac34
\end{align}

For $p_{6,6}$ we know that state 6 is in a closed communicating class hence it is recurrent so the probability
of returning is 1.
\newline 
\newline
\textbf{e)}
To solve $p_{1,4}$ we will define:
\begin{equation}
    V_{i} = P(\text{Starting at i and ending up in 4})
\end{equation}
\begin{align}
    V_1 &=\frac12V_1 + \frac12V_2\\
    V_2 &= \frac12V_2 + \frac12V_3\\
    V_3 &= \frac13V_1 + \frac13V_3 +\frac13\\
\end{align}
Therefore with algebra and solving for $V_1$
\begin{align}
    V_1 &=V_2\\
    V_2 &= V_3\\
    V_3 &= \frac12V_1+\frac12\\
    \implies V_2 &= \frac12V_1 + \frac12\\
    \implies V_1 &= \frac12V_1 + \frac12\\
    \frac12V_1 &= \frac12\\
    \implies p_{1,4} = V_1 &= 1\\
\end{align}
For $p_{5,1}$ the probability of this is occuring is 0 as state 5 is in a closed class that does not
contain 1.

\section{Markov Chain 3}
This is not a markov chain. X can take on 3 states specifically 0,1,2. 
We will show through contradiction:
\begin{align}
    P(X_{n+1} = 2| X_n = 1) &= \frac{P(X_{n+1} = 2,X_n = 1)}{P(X_n = 1)}\\
    P(X_{n+1} = 2,X_n = 1) &= P(Y_{n+1} = 1,Y_n =1,Y_{n-1} = 0) = 1/8\\
    P(X_{n} = 1) &= P(Y_n = 1|Y_{n-1} = 0)P(Y_{n-1} = 0) + P(Y_n = 0|Y_{n-1} = 1)P(Y_{n-1} = 1) = 1/2\\
    \implies P(X_{n+1} = 2| X_n = 1) &= 1/4
\end{align}
However:
\begin{equation}
    P(X_{n+1} = 2| X_n = 1, X_{n-1}=2) = \frac{P(X_{n+1} = 2,X_n = 1, X_{n-1}=2)}{P(X_n = 1, X_{n-1}=2)} = 0
\end{equation}
This is because if $X_{n-1} = 2$ and $X_n = 1$ then $Y_{n-1} = 1, Y_{n-2} = 1$ which forces $Y_n = 0$ to satisfy
$X_n$. This means $Y_n+Y_{n+1} = X_{n+1} \leq 2$

Subsequently:
\begin{equation}
    0 = P(X_{n+1} = 2| X_n = 1, X_{n-1}=2) \neq P(X_{n+1} = 2| X_n = 1) = 1/4
\end{equation}
The markovian property is not satisfied hence the sequence $X_1,X_2,X_3,...$ is not a markov chain.
\end{document}